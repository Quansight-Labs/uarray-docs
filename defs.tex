
\usepackage{amsmath,amssymb,wasysym}
\usepackage[margin=2cm]{geometry}

\usepackage{comment}
\usepackage{longtable}
\usepackage{tabularx,booktabs}
\usepackage{listings}
\lstset{basicstyle=\ttfamily}
\newcommand{\py}{\lstinline}

\newcommand{\dims}{\delta\,}
\newcommand{\shape}{\rho\,}
\newcommand{\reshape}{\,\rho\,}
\newcommand{\size}{\tau\,}
\newcommand{\product}{\pi\,}
\newcommand{\ravel}{\mathrm{rav}\,}
\newcommand{\take}[2]{{#1\,\uparrow\,#2}}
\newcommand{\drop}[2]{{#1\,\downarrow\,#2}}
\newcommand{\minus}[1]{{}^{\boldsymbol{\mbox{-}}\!}{#1}}
\newcommand{\rop}[1]{\,\mathrm{#1}^*\,}
\newcommand{\op}[1]{\,\mathrm{#1}\,}
\newcommand{\uop}[1]{\mathrm{#1}\,}
\newcommand{\id}[1]{\mathrm{id}(\op{#1})}
\newcommand{\red}[1]{{}_{\op{#1}\!}\mathrm{red}\,}
\newcommand{\scan}[1]{{}_{\op{#1}\!}\mathrm{scan}\,}
\newcommand{\gu}{\mathrm{gu}\,}
\newcommand{\gd}{\mathrm{gd}\,}
\newcommand{\range}{\iota\,}
\newcommand{\compress}{\,\notslash\,}
\newcommand{\expand}{\,\notbackslash\,}
\newcommand{\reverse}{\phi\,}
\newcommand{\rotate}[1]{{#1}\theta\,}
\newcommand{\transpose}[1]{{#1}\bigcirc\!\!\!\!\!\!\backslash\,\,} % find proper symbol
\newcommand{\hop}[2]{{{}_{#1}\Omega_{#2}}\,}
\newcommand{\getitem}[2]{{#2}\,\psi\,{#1}}

\newcommand{\Z}{{\mathbb{Z}}}

\title{MoA formalism and arrays in software}
\author{Pearu Peterson}

\begin{document}
\maketitle
\section{Introduction}

MoA - Mathematics of Arrays - defines formalism (based on APL) to
describe multidimensional arrays and operations on these. On the other
hand, contemporary array processing software (NumPy, Xnd, Apache
Arrow, etc) implement multidimensional array container objects and
define UI to interact with the array objects. The aim of this document
is to provide a mapping between MoA formalism and UI-s of widely used
array processing software.



\section{Basic definitions}

The MoA formalism is originally defined in ``A Mathematics of Arrays''
by Lenore Mullin, PhD thesis, 1988.  We use NumPy UI as a
representative of a software implementing UI for multidimensional
arrays. For brevity, we assume \verb+from numpy import *+.

In general, \emph{an array is an object that has a mapping to get
  array items} \footnote{A mapping $f$ is a set of argument and value
  pairs,$(i,v)$, and is represented as $f(i)=v$. We consider only
  single-valued mappings.} . The arguments to the mapping are called
\emph{indices}, all valid indices define the so-called \emph{index
  set} of an array. The items can be arbitrary objects.

Two arrays are \emph{equal} iff their get items mappings are equal.

\emph{Element-wise operations} of two arrays are defined iff the index
sets of the two arrays are equal.

Arrays can be classified according to how the mappings are defined:
\begin{description}
\item[Multidimensional arrays:] The index set of
  an $N$-dimensional array is an $N$-dimensional hyperrectangle in
  $\Z^N$: $[0,\ldots,s_0-1]\times\cdots\times[0,\ldots,s_{N-1}-1]$
  where non-negative integers $(s_0,\ldots,s_{N-1})$ define the shape
  of an array.
\item[Ragged arrays:] The index set of an
  $N$-dimensional ragged array is a connected subset of an
  $N$-dimensional hyperrectangle in $\Z^N$. The concrete
  representation of the index set may have different forms.
\item[Sparse arrays:] The index set of an sparse array is a possibly
  disconnected subset of $N$-dimensional hyperrectangle in $\Z^N$ and
  is extended to the full hyperrectangle by defining the default
  value.
\end{description}

The MoA deals with multidimensional arrays only.  The NumPy implements
array object for multidimensional arrays only. There is one-to-one
relation between MoA formalism and NumPy UI. Xnd implements a more
general container object than NumPy ndarray.

The goal is to extend MoA formalism to ragged arrays and then use this
to define an UI as well as optimized algorithms to the array object
implementation software such as Xnd. 

The current working hypothesis is that the ragged arrays can be
represented as arrays of arrays where the subarrays are
multidimensional arrays.

\section{Multidimensional arrays: MoA and NumPy}

\begin{comment}
\newcommand{\MoADefinition}[1]{
  \begin{itemize}
  #1
  \end{itemize}
}
\newcommand{\MoAItem}[3]{\item {#1}
  \begin{description}
  \item[MoA:] {#2}
  \item[NumPy:] {#3}
  \end{description}
}  
\end{comment}
%\begin{comment}
\newcommand{\MoADefinition}[1]{
\renewcommand{\arraystretch}{2}
\begin{longtable}{| p{.5\textwidth} | p{.5\textwidth} |}
#1
\end{longtable}
\renewcommand{\arraystretch}{1}
}
\newcommand{\MoAItem}[3]{
\multicolumn2{l}{
  \begin{minipage}[l]{0.9\linewidth} \vspace{2ex}
  $\bullet$ #1\\[0.1ex]
  \end{minipage}
}\\\hline
\begin{minipage}[l]{0.95\linewidth}
{#2}  
\end{minipage}
&
\begin{minipage}[l]{0.95\linewidth}
{#3}  
\end{minipage}\\\hline
}
%\end{comment}




We have the following correspondence table for basic notions:
\MoADefinition{
  \MoAItem{a multidimensional array}{$\xi$}{\py'x=array(...)'}
  \MoAItem{a one dimensional array, a vector}{$\vec v\equiv<v_0\, v_1\, \ldots>$}{\py'v=array([v_0, v_1, ...])'}
  \MoAItem{total number of array items}{$\size\xi$}{\py'x.size'}
  \MoAItem{the shape of an array is a vector of non-negative integers}{$\shape\xi\equiv<s_0\, \ldots\, s_{n-1}>$}{\py`x.shape==(s_0,...,s_n1)`}
  \MoAItem{the dimensionality of an array is non-negative integer}{$\dims\xi\equiv n\equiv\size\shape\xi$}{\py'x.ndim==n==len(x.shape)'}
  \MoAItem{an array item}{$\getitem\xi{\vec i}\equiv\xi[i_0;\ldots;i_{n-1}]$}{\py'x[i_0,...,i_n1]'}
  \MoAItem{negative of $k$}{$\minus k$}{\py'-k'}
  \MoAItem{$\gamma$ function is used to define the mapping between the indices of an array and its ravelled version}{
  \begin{eqnarray*}
\gamma(<>;<>)&\equiv&0\\
\gamma(\vec a;\vec b)&\equiv&a_{-1} + b_{-1}\gamma(\drop{\minus1}{\vec a}; \drop{\minus1}{\vec b})
  \end{eqnarray*}
}{}
  \MoAItem{inverse of $\gamma$ with respect to the first argument}{$\gamma'\ldots \gamma(\gamma'(n;\vec x); \vec x)=n$}{}
}

MoA defines a number of operators on multidimensional arrays:
\MoADefinition{
  \MoAItem{collapse an array to one dimension (row ordering is assumed)}{$\ravel \xi$}{\py'x.ravel()'}
  \MoAItem{product of vector items}{$\product\vec v$}{\py'v.prod()'}

}

MoA defines array construction operators:
\MoADefinition{
  \MoAItem{join a sequences of vectors}{$\vec v,\vec w,\ldots$}{\py'concatenate((v, w, ...))'}
  \MoAItem{a range of integers}{$\range n\equiv 0,1+\range(n-1)$}{\py'arange(n)'}
  \MoAItem{compress}{${\vec v}_l\compress{\vec v}_r$}{\py''}
  \MoAItem{}{${\vec v}_l\expand{\vec v}_r$}{\py''}
  % \MoAItem{}{$$}{\py''}
  % \MoAItem{}{$$}{\py''}
}


MoA defines various slicing operations:
\MoADefinition{
  \MoAItem{take first $k$ items; take last $k$ items}{$\take k{\vec v}\,;\,\take{\minus k}{\vec v} $}{\py'v[:k] ; v[-k:]'}
  \MoAItem{drop first $k$ items; drop last $k$ items}{$\drop k{\vec v}\,;\,\drop{\minus k}{\vec v} $}{\py'v[k:] ; v[:-k]'}
  \MoAItem{}{$\take\sigma\xi$}{\py''}
  \MoAItem{}{$\drop\sigma\xi$}{\py''}
  \MoAItem{}{$\take{\vec v}\xi$}{\py''}
  \MoAItem{}{$\drop{\vec v}\xi$}{\py''}
}

MoA defines the following relational operators,$(\rop{R}, \text{\py'rop'}) \in \{(\rop{>}, \text{\py'gt'}), (\rop{<}, \text{\py'lt'}), (\rop{=}, \text{\py'eq'}), \ldots\}$ :
\MoADefinition{
  \MoAItem{element-wise relational operators}{$\xi_l\rop{R}\xi_r$}{\py'x_l.__<rop>__(x_r).all()'}
  \MoAItem{scalar extension of relational operators}{$\xi\rop{R}a$}{\py'x.__<rop>__(a*ones(x.shape)).all()'}
  \MoAItem{}{$a\,R^{*}\xi$}{\py''}
}

MoA defines the following binary and unary operators, $(\op{op},\text{\py'op'})\in\{
(\op+, \text{\py'add'}), (\op-, \text{\py'sub'}), (\op\times, \text{\py'mul'}), \ldots
\}$:
\MoADefinition{
  \MoAItem{element-wise binary operators}{$\xi_l\op{op}\xi_r$}{\py'x_l.__<op>__(x_r)'}
  \MoAItem{scalar extension of binary operators}{$\xi\op{op}a\quad;\quad a\op{op}\xi$}{\py'x.__op__(a*ones(x.shape))'}
  \MoAItem{scalar extension of unary operators}{$\uop{uop}\xi$}{\py'x.__<uop>__()'}
}

MoA defines various array transformation operators:
\MoADefinition{
  \MoAItem{reshape a vector}{$\vec s\reshape\vec v$}{\py'v.reshape(s)'}
  \MoAItem{reverse vector items}{$\reverse \vec v$}{\py'v[::-1]'}
  \MoAItem{}{$\rotate{\sigma}\xi$}{\py''}
  \MoAItem{transpose of an array}{$\transpose{\vec v}\xi$}{\py''}
  \MoAItem{reduction}{$\red{op}\xi\equiv(\getitem\xi{<0>})\op{op}\red{op}(\drop1\xi)$}{}
}


TODO:\\
\begin{tabularx}{\textwidth}{lX}
\begin{tabular}[m]{m{0.2\linewidth}|l|m{0.45\linewidth}|}
  MoA & Python & Description \\\hline
$\vec i\psi\xi;\qquad \size\vec i <\size\shape\xi$ & \verb|x[i+(Ellipsis,)]| & slice of an array\\
$\xi_l\psi\xi_r$ &&\\
\multicolumn2{l|}{$\id{op}\ldots \sigma\op{op}\id{op}\equiv\sigma$}& identity of an operator $\op{op}$\\
$\scan{op}$ && \\
$\xi_l,\xi_r$ & & \\
$\xi,\sigma\equiv \xi, (1,(\drop1{\shape\xi}))\shape\sigma$ &&\\
$\vec v[\vec x]$ &&\\
$\gu\vec v\,;\,\gd\vec v$ &&\\
$\hop{f}{\vec a}\xi$&&\\
$\xi_l(\hop{f}{\vec a})\xi_r$&&\\

\end{tabular}  
\end{tabularx}


