Let $\xi$ denote an array in MoA formalism
and \verb|x| be the NumPy array in Python. Then we have the following
correspondence table:\\[2ex]
\begin{tabularx}{\textwidth}{lX}
\begin{tabular}[m]{l|l|m{0.45\linewidth}|}
  MoA & Python & Description \\\hline
  $\xi$ & \verb|x=array(...)| 
               & a multidimensional array\\
  $\vec v\equiv<a\, b\, c>$ & \verb|v=array([a, b, c])| 
               & a one dimensional array, a vector\\
  $\dims\xi\equiv n$ & \verb|x.ndim==n| 
               & the dimensionality of an array \\
  $\shape\xi\equiv<s_0\, \ldots\, s_{n-1}>$ & \verb|x.shape==(s_0,...,s_n1)| 
               & the shape of an array \\
  $\vec i\psi\xi\equiv\xi[i_0;\ldots;i_{n-1}]$ & \verb|x[i_0,...,i_n1]| 
               & an array item\\
  $\ravel \xi$ & \verb|x.ravel()| 
               & collapse an array to one dimension (row ordering is assumed)\\
  $\product\vec v$ & \verb|x.prod()| 
               & product of vector items\\
  $\size\xi$   & \verb|x.size| 
               & total number of array items\\
  $\vec v,\vec w,\ldots$ & \verb|concatenate((v, w, ...))|
               & join sequences of vectors\\
  $\minus k$ & \verb|-k| & negative of scalar value $k$\\
  $\take k{\vec v}; \take{\minus k}{\vec v} $&\verb|v[:k] ; v[-k:]|
               & take first $k$ items; take last $k$ items\\

                                                           
\end{tabular}  
\end{tabularx}
